% Copyright (c) 2014,2016,2018 Casper Ti. Vector
% Public domain.

\chapter{引言}

\section{研究背景}
这里写点研究背景吧。如图\ref{fig:timeline}是latex画框图的示例,
可以用AI辅助生成,或者用visio/ppt等工具画出来再插入图片。

\begin{figure}[htbp]
    \centering
    \begin{tikzpicture}[
        timeline/.style={draw, thick},
        stage/.style={rectangle, draw=blue!50, fill=blue!10, thick, text width=6.5cm, minimum height=2.3cm, align=left, rounded corners},
        year/.style={font=\bfseries},
        node distance=1.2cm
    ]
    
    % 时间轴竖线
    \draw[timeline, -{Latex[length=3mm]}] (0,0) -- (0,-12);
    
    % 时间点及标记
    \foreach \y/\year in {0/1960s末, -4/2000, -8/2025, -12/今}
        \draw[thick] (-0.2,\y) -- (0.2,\y) node[left=10pt, year] {\year};
    
    % 阶段一
    \node[stage, anchor=west] at (2,-1.8) {
      \textbf{阶段一:1} \\
      \textbf{时间:} 1960s末—2000年 \\
      \textbf{代表项目:} 旧的 \\
      \textbf{特点:} 测试
    };
    
    % 阶段二
    \node[stage, anchor=west] at (2,-5.8) {
      \textbf{阶段二:2} \\
      \textbf{时间:} 2000—2025年 \\
      \textbf{代表项目:} 稍旧的 \\
      \textbf{特点:} 测试
    };
    
    % 阶段三
    \node[stage, anchor=west] at (2,-9.8) {
      \textbf{阶段三:3} \\
      \textbf{时间:} 2025年至今 \\
      \textbf{代表项目:} 新的 \\
      \textbf{特点:} 测试
    };
    
    \end{tikzpicture}
    \caption{项目背景发展阶段时间轴图}
    \label{fig:timeline}
\end{figure}

\section{研究意义}
\label{sec:research-significance}
随便写写。加一个引用吧。\supercite{test-zh}。