% Copyright (c) 2014,2016,2021 Casper Ti. Vector
% Public domain.

\begin{cabstract}
	这是摘要
\end{cabstract}

\ifblind\begin{beabstract}\else\begin{eabstract}\fi
	This is the abstract.
\ifblind\end{beabstract}\else\end{eabstract}\fi

% vim:ts=4:sw=4
